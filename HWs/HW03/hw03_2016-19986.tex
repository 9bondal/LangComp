\documentclass[a4paper,11pt]{scrartcl}
\usepackage{mmap}
\usepackage[margin=1in]{geometry} % decreases margins
\usepackage{setspace}
\setlength{\parskip}{0pt}
\onehalfspacing

\usepackage{enumitem}
\setlist{noitemsep, topsep=0pt, partopsep=0pt, leftmargin=*}

\usepackage{textcomp}
\usepackage{booktabs}
\usepackage[mathrm=sym]{unicode-math}

\usepackage[final]{hyperref} % adds hyper links inside the generated pdf file
\hypersetup{
  colorlinks=true,% false: boxed links; true: colored links
  linkcolor=blue,        % color of internal links
  citecolor=blue,   % color of links to bibliography
  filecolor=magenta,     % color of file links
  urlcolor=blue
}
\urlstyle{rm}

\usepackage[procnames]{listings}
\usepackage[dvipsnames]{xcolor}

\makeatletter
\lstdefinestyle{mystyle}{
  language=python,
  showstringspaces=false,
  otherkeywords={__eq__,__setitem__,__abs__,wrong_int},
  basicstyle=%
    \ttfamily
    \lst@ifdisplaystyle\normalsize\fi,
  keywordstyle=\color{Blue},
  commentstyle=\color[gray]{0.6},
  stringstyle=\color[RGB]{233,125,44},
  procnamekeys={class},
  procnamestyle=\color{Bittersweet},
  emph={None,True,False},
  emphstyle=\itshape\color[rgb]{0.7,0,0},
  emph={[2]self},
  emphstyle=[2]\itshape\color{Bittersweet}
}
\makeatother

\lstset{style=mystyle}
\lstset{escapeinside={(*@}{@*)}}

\usepackage{fontspec}
\usepackage{bold-extra}
\setmonofont[AutoFakeSlant=0.2,Scale=0.95]{D2Coding}
\setsansfont[BoldFont=AppleSDGothicNeo-SemiBold]{Apple SD Gothic Neo}
\setmainfont[AutoFakeSlant=0.2,BoldFont=SDMyeongjoNeoa-eSm,WordSpace={1.0,0.5,0.5},Kerning=On]{SDMyeongjoNeoa-bLt}

\usepackage{kotex}

\addtokomafont{labelinglabel}{\bfseries}
\addtokomafont{title}{\bfseries}

\setkomafont{disposition}{\normalfont}
\setkomafont{section}{\LARGE\bfseries\sffamily}
\setkomafont{subsection}{\Large\mdseries\sffamily}

\title{\vspace{-0.5in}LangComp HW3}
\author{\vspace{-15pt}2016-19986 정누리}
\date{\vspace{-5pt}\today}

%++++++++++++++++++++++++++++++++++++++++

\begin{document}

\maketitle

\setcounter{section}{1}
\section{Flow Control}

\begin{labeling}{Q10}
  \item[Q1]
  \lstinline{True}와 \lstinline{False}. 왼쪽과 같이 적는다.

  \item[Q2]
  \lstinline{and}, \lstinline{or}, \lstinline{not}.

  \item[Q3] 각각 아래와 같다.
  \begin{center}
    \begin{tabular}{cc}
      \begin{tabular}[t]{ c c c }
        \toprule
        \(P\)          & \(Q\)          & \(P \wedge Q\) \\
        \midrule
        \(\mathrm{T}\) & \(\mathrm{T}\) & \(\mathrm{T}\) \\
        \(\mathrm{T}\) & \(\mathrm{F}\) & \(\mathrm{F}\) \\
        \(\mathrm{F}\) & \(\mathrm{T}\) & \(\mathrm{F}\) \\
        \(\mathrm{F}\) & \(\mathrm{F}\) & \(\mathrm{F}\) \\
        \bottomrule
      \end{tabular}
      \qquad
      \begin{tabular}[t]{ c c c }
        \toprule
        \(P\)          & \(Q\)          & \(P \vee Q\)   \\
        \midrule
        \(\mathrm{T}\) & \(\mathrm{T}\) & \(\mathrm{T}\) \\
        \(\mathrm{T}\) & \(\mathrm{F}\) & \(\mathrm{T}\) \\
        \(\mathrm{F}\) & \(\mathrm{T}\) & \(\mathrm{T}\) \\
        \(\mathrm{F}\) & \(\mathrm{F}\) & \(\mathrm{F}\) \\
        \bottomrule
      \end{tabular}
      \qquad
      \begin{tabular}[t]{ c c }
        \toprule
        \(P\)          & \(\neg P\)     \\
        \midrule
        \(\mathrm{T}\) & \(\mathrm{F}\) \\
        \(\mathrm{F}\) & \(\mathrm{T}\) \\
        \bottomrule
      \end{tabular}
    \end{tabular}
  \end{center}

  \item[Q4]
  \begin{lstlisting}
(5 > 4) and (3 == 5)                -> False
not (5 > 4)                         -> False
(5 > 4) or (3 == 5)                 -> True
not ((5 > 4) or (3 == 5))           -> False
(True and True) and (True == False) -> False
(not False) or (not True)           -> True
  \end{lstlisting}

  \item[Q5]
  \lstinline{==}, \lstinline{!=}, \lstinline{<}, \lstinline{<=}, \lstinline{>}, \lstinline{>=}

  \item[Q6]
  Assignment operator assigns a value or an object reference to the left-hand operand\footnote{This is just a simple assignment if the left-hand operand is a simple variable, but for classes' and more complex objects' fields, or for mutable, subscriptable objects, the behavior might be different. For example, list element assignment will call \lstinline|__setitem__| special method.}, equal to operator calls the special method \lstinline{__eq__}\footnote{Operands are chosen based on types (\url{https://docs.python.org/3/reference/datamodel.html\#object.__eq__}).}. If no such implementation is defined, it falls back to the default behavior of python \lstinline{object.__eq__} (which is based on object identity\footnote{\url{https://docs.python.org/3/reference/expressions.html\#value-comparisons}}).

  \item[Q7]
  \emph{Conditions} are special expressions that always evaluate down to a boolean value. A \linebreak flow control statement will decide what to do based on the \emph{conditions}.

  \item[Q8]
  Python determines block by whitespaces. Assuming that the whitespaces are all tabs or all spaces, the three blocks are:

  \begin{lstlisting}
if spam == 10:
    {first block}
    if spam > 5:
        {second block}
    else:
        {third block}
   (*@ $\cdots$ @*)
  \end{lstlisting}

  \item[Q9]
  \begin{lstlisting}
if spam == 1:
    print('Hello')
elif spam == 2:
    print('Howdy')
else:
    print('Greetings!')
  \end{lstlisting}

  \item[Q10]
  \lstinline{ctrl + c}.

  \item[Q11]
  \lstinline{break} escapes loop, \lstinline{continue} skips the rest of current loop, checks condition again, then jumps to the next loop or escapes loop depending on the condition.

  \item[Q12]
  All three are the same.

  \item[Q13]
  First program:
  \begin{lstlisting}
for i in range(1, 11):
    print(i)
  \end{lstlisting}
  Second program:
  \begin{lstlisting}
i = 0
while i < 10:
    i += 1
    print(i)
  \end{lstlisting}

  \item[Q14]
  \begin{lstlisting}
import spam
spam.bacon()
  \end{lstlisting}
\end{labeling}

\begin{labeling}{Extra}
  \item[Extra]
  Official document says\footnote{\url{https://docs.python.org/3/library/functions.html\#round}}:

  \begin{quotation}
    \lstinline{round(number[, ndigits])}: Return \emph{number} rounded to \emph{ndigits} precision after the decimal point. If \emph{ndigits} is omitted or is \lstinline{None}, it returns the nearest integer to its input. \dots rounding is done toward the even choice \dots
  \end{quotation}

  \lstinline{round()} function에 대한 실험은 지난 과제에서 이미 진행하였으므로 생략하도록 하겠다.

  이제 \lstinline{abs()} function에 대해 살펴보자\footnote{\url{https://docs.python.org/3/library/functions.html\#abs}}.

  \begin{quotation}
    Return the absolute value of a number. The argument may be an integer or a floating point number. If the argument is a complex number, its magnitude is returned. If \emph{x} defines \lstinline{__abs__()}, \lstinline{abs(x)} returns \lstinline{x.__abs__()}.
  \end{quotation}

  \lstinline{abs(x)}는 내부적으로 \lstinline{__abs__} magic method를 호출하는 것을 확인할 수 있다. 따라서 built-in type에 대해서는 \lstinline{abs()}가 수학적으로 올바른 결과를 낼 것으로 기대할 수 있으나, custom class에 대해 사용했을 때 항상 옳은 결과를 낼 것이라 기대하기는 어렵다.

  예를 들어, \lstinline{int} class를 상속받는 다음 \lstinline{wrong_int} class를 생각해 보자.

  \begin{lstlisting}
    class wrong_int(int):
        def __abs__(self):
            return self
  \end{lstlisting}

  이제 해당 class를 이용하여 만든 object에 \lstinline{abs()}를 취하면 일반적으로 기대되는 결과가 얻어지지 않는다.

  \begin{lstlisting}
    >> i = wrong_int(-10)
    >> i
    -10
    >> abs(i)
    -10
  \end{lstlisting}

  물론 다시 \lstinline{int}로 캐스팅하면 정상적인 결과를 얻는다.

  \begin{lstlisting}
    >> i = int(i)
    >> i
    -10
    >> abs(i)
    10
  \end{lstlisting}

  이러한 현상은 모든 대상이 object로 취급되고 (Python 3), duck typing 언어인 만큼 많은 built-in function 및 operator가 object에 구현을 delegate하는 python의 특징으로도 볼 수 있겠다. 이러한 delegation은 function을 first-class object로 보며, parameter list에 대한 정보를 runtime에 얻는 파이썬의 특징에서 생기는 부작용인 function overloading 불가능을 해결하기 위한 방법이기도 하다.

  반면 normal typing인 언어\footnote{Java 등이 대표적이다.}는 이러한 형태의 function은 만들기 어렵다. Generic programing을 지원하는 static typed 언어 중, C++ 등의 일부 언어는 특이하게도 python과 유사한 코드를 duck typing 언어가 아님에도 작성할 수 있는데, 이는 compile time에 컴파일러가 각각의 타입에 맞는 함수를 생성해 주기 때문이다. 일종의 편의기능에 가까운 셈인데, 따라서 python처럼 runtime에도 duck typing으로 동작하지는 않는다.
\end{labeling}

\newpage
\setcounter{section}{3}
\section{Lists}

\begin{labeling}{Q10}
  \item[Q1]
  An empty list.

  \item[Q2]
  \lstinline{spam[2] = 'hello'}.

  \item[Q3]
  \begin{lstlisting}
spam[int(int('3' * 2) // 11)] -> spam[int(int('33') // 11)]
                              -> spam[int(33 // 11)]
                              -> spam[3] -> 'd'
  \end{lstlisting}

  \item[Q4]
  \lstinline{'d'}.

  \item[Q5]
  \lstinline{['a', 'b']}.

  \item[Q6]
  \lstinline{1}.

  \item[Q7]
  \lstinline{[3.14, 'cat', 11, 'cat', True, 99]}.

  \item[Q8]
  \lstinline{[3.14, 11, 'cat', True]}.

  \item[Q9]
  \lstinline{+} and \lstinline{*}, respectively.

  \item[Q10]
  \lstinline{append()} always put value at the end of the list, while \lstinline{insert()} put value at the specified position.

  \item[Q11]
  \lstinline{remove()} method or \lstinline{del} keyword.

  \item[Q12]
  Indexing and slicing operations are available. Both can be used with \lstinline{for} loops, \lstinline{len()} function, and membership tests (\lstinline{in} or \lstinline{not in}).

  \item[Q13]
  Lists are mutable, tuples are immutable.

  \item[Q14]
  \lstinline{(42,)}

  \item[Q15]
  \lstinline{tuple(some_list)}, \lstinline{list(some_tuple)}.

  \item[Q16]
  They contain list reference (or a \emph{pointer} to the list).

  \item[Q17]
  \lstinline{copy.copy()} copies the top-most list object (shallow copy), which might contain object references. \lstinline{copy.deepcopy()} copies all contained objects recursively (`deep' copy).

\end{labeling}

\end{document}
