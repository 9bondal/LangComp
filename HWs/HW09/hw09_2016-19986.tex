\documentclass[a4paper,11pt]{scrartcl}
\usepackage[margin=1in]{geometry} % decreases margins
\usepackage{setspace}
\setlength{\parskip}{0pt}
\onehalfspacing

\usepackage{enumitem}
\setlist{noitemsep, topsep=0pt, partopsep=0pt, leftmargin=*}

\usepackage{textcomp}
\usepackage{booktabs}
\usepackage[mathrm=sym]{unicode-math}

\usepackage[final]{hyperref} % adds hyper links inside the generated pdf file
\hypersetup{
  colorlinks=true,% false: boxed links; true: colored links
  linkcolor=blue,        % color of internal links
  citecolor=blue,   % color of links to bibliography
  filecolor=magenta,     % color of file links
  urlcolor=blue
}
\urlstyle{rm}

\usepackage[procnames]{listings}
\usepackage[dvipsnames]{xcolor}

\makeatletter
\lstdefinestyle{mystyle}{
  showstringspaces=false,
  basicstyle=%
    \ttfamily
    \lst@ifdisplaystyle\normalsize\fi,
  keywordstyle=\color{Blue},
  commentstyle=\color[gray]{0.6},
  stringstyle=\color[RGB]{233,125,44},
  procnamekeys={class},
  procnamestyle=\color{Bittersweet},
  emph={None,True,False},
  emphstyle=\itshape\color[rgb]{0.7,0,0},
  emph={[2]self},
  emphstyle=[2]\itshape\color{Bittersweet}
}
\makeatother

\lstset{style=mystyle}
\lstset{escapeinside={(*@}{@*)}}

\usepackage{fontspec}
\usepackage{bold-extra}
\setmonofont[AutoFakeSlant=0.2,Scale=0.95]{D2Coding}
\setsansfont[BoldFont=AppleSDGothicNeo-SemiBold]{Apple SD Gothic Neo}
\setmainfont[AutoFakeSlant=0.2,BoldFont=SDMyeongjoNeoa-eSm,WordSpace={1.0,0.5,0.5},Kerning=On]{SDMyeongjoNeoa-bLt}

\usepackage{kotex}

\addtokomafont{labelinglabel}{\bfseries}
\addtokomafont{title}{\bfseries}

\setkomafont{disposition}{\normalfont}
\setkomafont{section}{\LARGE\bfseries\ttfamily}
\setkomafont{subsection}{\Large\mdseries\ttfamily}

\usepackage{indentfirst}
\usepackage{graphicx}

\title{\vspace{-0.5in}LangComp HW9}
\author{\vspace{-15pt}2016-19986 정누리}
\date{\vspace{-5pt}\today}

%++++++++++++++++++++++++++++++++++++++++

\begin{document}

\maketitle

\section*{vectors.py}

함수의 결과값을 \lstinline{print()} 함수로 출력하도록 한 뒤 실행한 결과는 아래와 같다.

\begin{lstlisting}
[4, 11, 11, 12]
[ 4 11 11 12]
[-2, 3, 5, -6]
[-2  3  5 -6]
[4, 20, 13, 17]
[ 4 20 13 17]
[-2, -14, -16, -6]
[ -2 -14 -16  -6]
[1.3333333333333333, 6.666666666666666,
 4.333333333333333, 5.666666666666666]
[1.33333333 6.66666667 4.33333333 5.66666667]
82
82
123
123
11.090536506409418
11.090536506409418
74
74
8.602325267042627
8.602325267042627
8.602325267042627
\end{lstlisting}

모두 결과는 동일하지만, 출력 형식에 약간의 차이가 있으며, \lstinline{numpy}의 경우 함수를 직접 정의할 필요 없이 기존에 정의되어 있는 것을 불러와 사용할 수 있다는 편의성이 있다.

재미있는 점은 \lstinline{%timeit} 커맨드를 이용해 마지막의 distance 계산 코드의 실행 시간을 측정해 봤을 때, \lstinline{numpy}가 더 빠를 것이라는 예상과 달리 실제로는 python native 함수의 실행 시간이 더 짧았다는 것이다. 각각 \lstinline{2.86 µs ± 76.8 ns} (native), \lstinline{17.9 µs ± 749 ns} (\lstinline{numpy}), \lstinline{28.5 µs ± 1.62 µs} (\lstinline{scipy})가 걸렸는데, \lstinline{list}를 \lstinline{np.ndarray}로 바꾸는 시간을 제외하여도 각각 약 10초와 21초로 더 느렸다. 이는 계산 시간이 매우 짧은 소규모 데이터 특성상 \lstinline{python}과 \lstinline{C} 사이에서 데이터가 이동하는 데 걸리는 시간이 계산 시간보다 더 길기 때문일 것이다.

\section*{statistics.py}

출력 결과는 다음과 같다. (import 과정에서 \lstinline{vectors.py}의 코드도 실행되는데, 해당 출력은 제외하고 첨부하였다.)

\begin{lstlisting}
7.333333333333333
7.333333333333333
6.0
6.0
[6, 1]
ModeResult(mode=array([1]), count=array([22]))
1
3
9
13
1.0
3.0
9.0
13.0
99
99
[92.66666666666667, 41.666666666666664,
  33.666666666666664, 32.666666666666664,
  17.666666666666668, 13.666666666666668,
  13.666666666666668, 11.666666666666668,
  11.666666666666668, 10.666666666666668,
  10.666666666666668, 8.666666666666668,
  7.666666666666667, 7.666666666666667,
  7.666666666666667, 7.666666666666667,
  6.666666666666667, 6.666666666666667,
  5.666666666666667, 5.666666666666667,
  5.666666666666667, 5.666666666666667,
  4.666666666666667, 4.666666666666667,
  ...
  -6.333333333333333, -6.333333333333333]
[92.66666667 41.66666667 33.66666667 32.66666667
  17.66666667 13.66666667 13.66666667 11.66666667
  11.66666667 10.66666667 10.66666667  8.66666667
  7.66666667  7.66666667  7.66666667  7.66666667
  6.66666667  6.66666667  5.66666667  5.66666667
  5.66666667  5.66666667  4.66666667  4.66666667
  ...
  -6.33333333 -6.33333333 -6.33333333 -6.33333333]
81.54351395730716
81.54351395730707
9.03014473623248
9.030144736232474
6
6.0
22.425435139573064
22.425435139573054
\end{lstlisting}

\begin{center}
  \includegraphics[scale=0.8]{Figure_1.png}
\end{center}

Mode의 경우만 출력이 다르고 나머지는 거의 동일했다. \lstinline{scipy.stats.mode()} 함수는 여러 개의 최빈값이 있는 경우 \href{https://docs.scipy.org/doc/scipy/reference/generated/scipy.stats.mode.html}{가장 작은 것만 돌려주기 때문}인데, 따라서 모든 최빈값을 알고 싶다면 해당 함수는 이용할 수 없다는 차이가 있겠다.

\end{document}
