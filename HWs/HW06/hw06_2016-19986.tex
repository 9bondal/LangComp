\documentclass[a4paper,11pt]{scrartcl}
\usepackage[margin=1in]{geometry} % decreases margins
\usepackage{setspace}
\setlength{\parskip}{0pt}
\onehalfspacing

\usepackage{enumitem}
\setlist{noitemsep, topsep=0pt, partopsep=0pt, leftmargin=*}

\usepackage{textcomp}
\usepackage{booktabs}
\usepackage[mathrm=sym]{unicode-math}

\usepackage[final]{hyperref} % adds hyper links inside the generated pdf file
\hypersetup{
  colorlinks=true,% false: boxed links; true: colored links
  linkcolor=blue,        % color of internal links
  citecolor=blue,   % color of links to bibliography
  filecolor=magenta,     % color of file links
  urlcolor=blue
}
\urlstyle{rm}

\usepackage[procnames]{listings}
\usepackage[dvipsnames]{xcolor}

\makeatletter
\lstdefinestyle{mystyle}{
  showstringspaces=false,
  basicstyle=%
    \ttfamily
    \lst@ifdisplaystyle\normalsize\fi,
  keywordstyle=\color{Blue},
  commentstyle=\color[gray]{0.6},
  stringstyle=\color[RGB]{233,125,44},
  procnamekeys={class},
  procnamestyle=\color{Bittersweet},
  emph={None,True,False},
  emphstyle=\itshape\color[rgb]{0.7,0,0},
  emph={[2]self},
  emphstyle=[2]\itshape\color{Bittersweet}
}
\makeatother

\lstset{style=mystyle}
\lstset{escapeinside={(*@}{@*)}}

\usepackage{fontspec}
\usepackage{bold-extra}
\setmonofont[AutoFakeSlant=0.2,Scale=0.95]{D2Coding}
\setsansfont[BoldFont=AppleSDGothicNeo-SemiBold]{Apple SD Gothic Neo}
\setmainfont[AutoFakeSlant=0.2,BoldFont=SDMyeongjoNeoa-eSm,WordSpace={1.0,0.5,0.5},Kerning=On]{SDMyeongjoNeoa-bLt}

\usepackage{kotex}

\addtokomafont{labelinglabel}{\bfseries}
\addtokomafont{title}{\bfseries}

\setkomafont{disposition}{\normalfont}
\setkomafont{section}{\LARGE\bfseries\sffamily}
\setkomafont{subsection}{\Large\mdseries\sffamily}

\title{\vspace{-0.5in}LangComp HW6}
\author{\vspace{-15pt}2016-19986 정누리}
\date{\vspace{-5pt}\today}

%++++++++++++++++++++++++++++++++++++++++

\begin{document}

\maketitle

\noindent 2.1번은 zero-width match를 허용하는지 여부에 따라 약간 달라지는데, zero-width match를 허용하지 않는다면 다음과 같다.

\begin{labeling}{2.2.4}
  \item[2.1.2] \texttt{/[a-zA-Z]+/}
  \item[2.1.2] \texttt{/[a-z]*b/}
  \item[2.1.3] 이 경우에는 \texttt{a} 한 문자만 존재하는 경우는 규칙에 맞지 않아 불가능하므로 b만 한 개 이상 존재하면 충분하다. 따라서 다음과 같다.

  \texttt{/(?:ba(?=b))*b+/}
\end{labeling}

\bigskip

\begin{labeling}{2.2.4}
  \item[2.2.1] 영문 알파벳이 아닌 나머지를 모두 word separator로 보면 다음과 같다.

  \lstinline{/([a-zA-Z]+)[^a-zA-Z]+\1/}

  \item[2.2.2] \lstinline{/^\d+.*\b[a-zA-Z]+$/}
  \item[2.2.3] \lstinline{/(?=.*\bgrotto\b)(?=.*\braven\b).*/}
  \item[2.2.4] 크게 다음 세 가지 경우로 나눌 수 있다.

    \begin{enumerate}[label=\theenumi)]
      \item 느낌표(\lstinline{!})나 물음표(\lstinline{?})로 끝나는 경우
      \item 마침표(\lstinline{.}), 느낌표 또는 물음표 뒤에 작은따옴표(\lstinline{'})나 큰따옴표(\lstinline{"})가 위치하는 경우
      \item 그리고 마침표로 끝나며, 바로 앞에 위치하는 단어가 \emph{abbreviation}이 아닌 경우
    \end{enumerate}

    특히 마지막이 상당히 까다로운데, 영어에 존재하는 모든 abbreviation을 포함하는 것은 불가능하므로 완벽히 거를 수는 없다. 다만 일반적으로 많이 쓰이는 줄임말을 negative lookbehind를 이용하여 작성하면 다음과 같다.

    \begin{lstlisting}
/(?:^|[!?]\s+|[.!?]['\"]\s+|
    (?<!Mr|Ms|Mrs|Dr|St|[Nn]o|[Vv]s|[Vv]\.s|
        [Ee]g|[Ee]\.g|[Pp][Ss]|[Pp]\.[Ss]|etc)\.\s+)
 ([a-zA-Z]+)/
    \end{lstlisting}
\end{labeling}

\end{document}
